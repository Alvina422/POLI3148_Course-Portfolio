% Options for packages loaded elsewhere
\PassOptionsToPackage{unicode}{hyperref}
\PassOptionsToPackage{hyphens}{url}
%
\documentclass[
]{article}
\usepackage{amsmath,amssymb}
\usepackage{iftex}
\ifPDFTeX
  \usepackage[T1]{fontenc}
  \usepackage[utf8]{inputenc}
  \usepackage{textcomp} % provide euro and other symbols
\else % if luatex or xetex
  \usepackage{unicode-math} % this also loads fontspec
  \defaultfontfeatures{Scale=MatchLowercase}
  \defaultfontfeatures[\rmfamily]{Ligatures=TeX,Scale=1}
\fi
\usepackage{lmodern}
\ifPDFTeX\else
  % xetex/luatex font selection
\fi
% Use upquote if available, for straight quotes in verbatim environments
\IfFileExists{upquote.sty}{\usepackage{upquote}}{}
\IfFileExists{microtype.sty}{% use microtype if available
  \usepackage[]{microtype}
  \UseMicrotypeSet[protrusion]{basicmath} % disable protrusion for tt fonts
}{}
\makeatletter
\@ifundefined{KOMAClassName}{% if non-KOMA class
  \IfFileExists{parskip.sty}{%
    \usepackage{parskip}
  }{% else
    \setlength{\parindent}{0pt}
    \setlength{\parskip}{6pt plus 2pt minus 1pt}}
}{% if KOMA class
  \KOMAoptions{parskip=half}}
\makeatother
\usepackage{xcolor}
\usepackage[margin=1in]{geometry}
\usepackage{color}
\usepackage{fancyvrb}
\newcommand{\VerbBar}{|}
\newcommand{\VERB}{\Verb[commandchars=\\\{\}]}
\DefineVerbatimEnvironment{Highlighting}{Verbatim}{commandchars=\\\{\}}
% Add ',fontsize=\small' for more characters per line
\usepackage{framed}
\definecolor{shadecolor}{RGB}{248,248,248}
\newenvironment{Shaded}{\begin{snugshade}}{\end{snugshade}}
\newcommand{\AlertTok}[1]{\textcolor[rgb]{0.94,0.16,0.16}{#1}}
\newcommand{\AnnotationTok}[1]{\textcolor[rgb]{0.56,0.35,0.01}{\textbf{\textit{#1}}}}
\newcommand{\AttributeTok}[1]{\textcolor[rgb]{0.13,0.29,0.53}{#1}}
\newcommand{\BaseNTok}[1]{\textcolor[rgb]{0.00,0.00,0.81}{#1}}
\newcommand{\BuiltInTok}[1]{#1}
\newcommand{\CharTok}[1]{\textcolor[rgb]{0.31,0.60,0.02}{#1}}
\newcommand{\CommentTok}[1]{\textcolor[rgb]{0.56,0.35,0.01}{\textit{#1}}}
\newcommand{\CommentVarTok}[1]{\textcolor[rgb]{0.56,0.35,0.01}{\textbf{\textit{#1}}}}
\newcommand{\ConstantTok}[1]{\textcolor[rgb]{0.56,0.35,0.01}{#1}}
\newcommand{\ControlFlowTok}[1]{\textcolor[rgb]{0.13,0.29,0.53}{\textbf{#1}}}
\newcommand{\DataTypeTok}[1]{\textcolor[rgb]{0.13,0.29,0.53}{#1}}
\newcommand{\DecValTok}[1]{\textcolor[rgb]{0.00,0.00,0.81}{#1}}
\newcommand{\DocumentationTok}[1]{\textcolor[rgb]{0.56,0.35,0.01}{\textbf{\textit{#1}}}}
\newcommand{\ErrorTok}[1]{\textcolor[rgb]{0.64,0.00,0.00}{\textbf{#1}}}
\newcommand{\ExtensionTok}[1]{#1}
\newcommand{\FloatTok}[1]{\textcolor[rgb]{0.00,0.00,0.81}{#1}}
\newcommand{\FunctionTok}[1]{\textcolor[rgb]{0.13,0.29,0.53}{\textbf{#1}}}
\newcommand{\ImportTok}[1]{#1}
\newcommand{\InformationTok}[1]{\textcolor[rgb]{0.56,0.35,0.01}{\textbf{\textit{#1}}}}
\newcommand{\KeywordTok}[1]{\textcolor[rgb]{0.13,0.29,0.53}{\textbf{#1}}}
\newcommand{\NormalTok}[1]{#1}
\newcommand{\OperatorTok}[1]{\textcolor[rgb]{0.81,0.36,0.00}{\textbf{#1}}}
\newcommand{\OtherTok}[1]{\textcolor[rgb]{0.56,0.35,0.01}{#1}}
\newcommand{\PreprocessorTok}[1]{\textcolor[rgb]{0.56,0.35,0.01}{\textit{#1}}}
\newcommand{\RegionMarkerTok}[1]{#1}
\newcommand{\SpecialCharTok}[1]{\textcolor[rgb]{0.81,0.36,0.00}{\textbf{#1}}}
\newcommand{\SpecialStringTok}[1]{\textcolor[rgb]{0.31,0.60,0.02}{#1}}
\newcommand{\StringTok}[1]{\textcolor[rgb]{0.31,0.60,0.02}{#1}}
\newcommand{\VariableTok}[1]{\textcolor[rgb]{0.00,0.00,0.00}{#1}}
\newcommand{\VerbatimStringTok}[1]{\textcolor[rgb]{0.31,0.60,0.02}{#1}}
\newcommand{\WarningTok}[1]{\textcolor[rgb]{0.56,0.35,0.01}{\textbf{\textit{#1}}}}
\usepackage{graphicx}
\makeatletter
\def\maxwidth{\ifdim\Gin@nat@width>\linewidth\linewidth\else\Gin@nat@width\fi}
\def\maxheight{\ifdim\Gin@nat@height>\textheight\textheight\else\Gin@nat@height\fi}
\makeatother
% Scale images if necessary, so that they will not overflow the page
% margins by default, and it is still possible to overwrite the defaults
% using explicit options in \includegraphics[width, height, ...]{}
\setkeys{Gin}{width=\maxwidth,height=\maxheight,keepaspectratio}
% Set default figure placement to htbp
\makeatletter
\def\fps@figure{htbp}
\makeatother
\setlength{\emergencystretch}{3em} % prevent overfull lines
\providecommand{\tightlist}{%
  \setlength{\itemsep}{0pt}\setlength{\parskip}{0pt}}
\setcounter{secnumdepth}{-\maxdimen} % remove section numbering
\ifLuaTeX
  \usepackage{selnolig}  % disable illegal ligatures
\fi
\IfFileExists{bookmark.sty}{\usepackage{bookmark}}{\usepackage{hyperref}}
\IfFileExists{xurl.sty}{\usepackage{xurl}}{} % add URL line breaks if available
\urlstyle{same}
\hypersetup{
  pdftitle={R markdown in class memo},
  pdfauthor={Alvina Chan},
  hidelinks,
  pdfcreator={LaTeX via pandoc}}

\title{R markdown in class memo}
\author{Alvina Chan}
\date{2023-09-21}

\begin{document}
\maketitle

\hypertarget{r-markdown}{%
\subsection{R Markdown}\label{r-markdown}}

\hypertarget{this-is-my-level-1-heading-2-hashes}{%
\subsection{This is my level 1 heading (2
hashes)}\label{this-is-my-level-1-heading-2-hashes}}

\hypertarget{this-is-my-level-2-heading-3-hashes}{%
\subsubsection{This is my level 2 heading (3
hashes)}\label{this-is-my-level-2-heading-3-hashes}}

\hypertarget{this-is-my-level-3-heading-4-hashes}{%
\paragraph{This is my level 3 heading (4
hashes)}\label{this-is-my-level-3-heading-4-hashes}}

\hypertarget{this-is-my-leve-4-heading-5-hashes}{%
\subparagraph{This is my leve 4 heading (5
hashes)}\label{this-is-my-leve-4-heading-5-hashes}}

This is my main text. POLI3148 is the course

\hypertarget{add-code-to-your-document-put-a-slash-then-there-will-be-options-then-select-r-code-chunk}{%
\section{\texorpdfstring{\textbf{Add code to your document (put a slash
then there will be options then select r code
chunk)}}{Add code to your document (put a slash then there will be options then select r code chunk)}}\label{add-code-to-your-document-put-a-slash-then-there-will-be-options-then-select-r-code-chunk}}

\begin{Shaded}
\begin{Highlighting}[]
\FunctionTok{log}\NormalTok{(}\DecValTok{50}\NormalTok{)}
\end{Highlighting}
\end{Shaded}

\begin{verbatim}
## [1] 3.912023
\end{verbatim}

\begin{Shaded}
\begin{Highlighting}[]
\DecValTok{1}\SpecialCharTok{+}\DecValTok{1}
\end{Highlighting}
\end{Shaded}

\begin{verbatim}
## [1] 2
\end{verbatim}

\hypertarget{output-figures}{%
\section{Output Figures}\label{output-figures}}

\begin{Shaded}
\begin{Highlighting}[]
\FunctionTok{plot}\NormalTok{(}\DecValTok{1}\SpecialCharTok{:}\DecValTok{10}\NormalTok{, }\DecValTok{1}\SpecialCharTok{:}\DecValTok{10}\NormalTok{)}
\end{Highlighting}
\end{Shaded}

\includegraphics{R-mark-down-in-cass-memo_files/figure-latex/unnamed-chunk-3-1.pdf}

\hypertarget{display-data-cars-data-as-test}{%
\section{Display data (cars data as
test)}\label{display-data-cars-data-as-test}}

\begin{Shaded}
\begin{Highlighting}[]
\NormalTok{cars}
\end{Highlighting}
\end{Shaded}

\begin{verbatim}
##    speed dist
## 1      4    2
## 2      4   10
## 3      7    4
## 4      7   22
## 5      8   16
## 6      9   10
## 7     10   18
## 8     10   26
## 9     10   34
## 10    11   17
## 11    11   28
## 12    12   14
## 13    12   20
## 14    12   24
## 15    12   28
## 16    13   26
## 17    13   34
## 18    13   34
## 19    13   46
## 20    14   26
## 21    14   36
## 22    14   60
## 23    14   80
## 24    15   20
## 25    15   26
## 26    15   54
## 27    16   32
## 28    16   40
## 29    17   32
## 30    17   40
## 31    17   50
## 32    18   42
## 33    18   56
## 34    18   76
## 35    18   84
## 36    19   36
## 37    19   46
## 38    19   68
## 39    20   32
## 40    20   48
## 41    20   52
## 42    20   56
## 43    20   64
## 44    22   66
## 45    23   54
## 46    24   70
## 47    24   92
## 48    24   93
## 49    24  120
## 50    25   85
\end{verbatim}

\hypertarget{use-tools-global-options-then-click-r-markdown-for-preview-in-pane-evaluate-by-project}{%
\section{Use Tools, Global options then click R Markdown for preview in
pane / evaluate by
project}\label{use-tools-global-options-then-click-r-markdown-for-preview-in-pane-evaluate-by-project}}

\hypertarget{inline-r-code}{%
\section{Inline R code}\label{inline-r-code}}

Scientific research shows that 12223+12111 =

\hypertarget{including-plots}{%
\subsection{Including Plots}\label{including-plots}}

You can also embed plots, for example:

\includegraphics{R-mark-down-in-cass-memo_files/figure-latex/pressure-1.pdf}

Note that the \texttt{echo\ =\ FALSE} parameter was added to the code
chunk to prevent printing of the R code that generated the plot.

install.packages(``tinytex'')

\begin{Shaded}
\begin{Highlighting}[]
\CommentTok{\#install.packages("tinytex")}
\CommentTok{\#tinytext::instal\textgreater{}tinytex()}
\CommentTok{\#\textasciigrave{}\textasciigrave{}\textasciigrave{}install.packages("tinytex")}
\CommentTok{\#tinytex::install\_tinytex()}


\CommentTok{\# \textasciigrave{}\textasciigrave{}\textasciigrave{}\{r\}}
\CommentTok{\#install.packages("tinytex")}
\CommentTok{\#tinytex::install\_tinytex()}
\end{Highlighting}
\end{Shaded}


\end{document}
